\documentclass[12pt]{article}%
\newcommand{\thisdocument}{}%
\newcommand{\thiscourse}{PSYC 3330: Elem Stats for Behav Sci}
\newcommand{\thissemester}{Week 7}
\usepackage{amsmath}%
\usepackage{amsfonts}%
\usepackage{amssymb}%
\usepackage{amsthm}%

%\textwidth = 6.5 in
%\textheight = 8 in
%\oddsidemargin = 0.0 in
%\evensidemargin = 0.0 in
%\topmargin = 0.0 in
%\headheight = 0.0 in
%\headsep = 0.0 in
\parskip = 0.1in
\parindent = 0.0in

\usepackage[left=1in,right=1in, top=1in, bottom=0.5in]{geometry}

\usepackage{fancyhdr,ifthen}
\pagestyle{fancy}
\lfoot{}  % no footers (in pagestyle fancy)
\cfoot{}
% running left heading
\lhead{\bfseries\Large\em \noindent{}\hspace{-.2em}\thiscourse{}\\[1mm]}
% running right heading
%\newcommand{\spc}{1.31em}
\newcommand{\spc}{1em}

\rhead{\em \sf Tarleton State University \hfill \sf \thisdocument{}}

\setlength{\headheight}{3ex}
\newcommand{\mainhead}[1]{\begin{center}{\Large \bf #1}\end{center}}
\newcommand{\head}[1]{\vspace{1.5ex}\par\noindent{\large \bf #1}\par\noindent}
\newcommand{\subhead}[1]{\vspace{2ex}\par\noindent{\sl #1}\vspace{1ex}\par\noindent{}}
\newcommand{\ptitle}{\sl}

\newcommand{\hra}{\hookrightarrow}


%%%% Theoremstyles
\theoremstyle{plain}
\newtheorem{theorem}{Theorem}[section]
\newtheorem{proposition}[theorem]{Proposition}
\newtheorem{corollary}[theorem]{Corollary}
\newtheorem{claim}[theorem]{Claim}
\newtheorem{lemma}[theorem]{Lemma}
\newtheorem{conjecture}[theorem]{Conjecture}

\theoremstyle{definition}
\newtheorem{definition}[theorem]{Definition}
\newtheorem{algorithm}[theorem]{Algorithm}
\newtheorem{question}[theorem]{Question}
\newtheorem{problem}[theorem]{Problem}
\newtheorem{goal}[theorem]{Goal}

\theoremstyle{remark}
\newtheorem{remark}[theorem]{Remark}
\newtheorem{remarks}[theorem]{Remarks}
\newtheorem{example}[theorem]{Example}
\newtheorem{exercise}[theorem]{Exercise}


\DeclareMathOperator{\SO}{SO}%
\DeclareMathOperator{\Sp}{Sp}%
\DeclareMathOperator{\SL}{SL}%
\DeclareMathOperator{\End}{End}%
\DeclareMathOperator{\Tr}{Tr}%
\DeclareMathOperator{\Res}{Res}%
\DeclareMathOperator{\res}{res}%
\DeclareMathOperator{\BSD}{BSD}%
\DeclareMathOperator{\Gal}{Gal}%
\DeclareMathOperator{\GL}{GL}%
\DeclareMathOperator{\Aut}{Aut}%
\DeclareMathOperator{\Reg}{Reg}%
\DeclareMathOperator{\Vis}{Vis}%
\DeclareMathOperator{\Ker}{Ker}%
\DeclareMathOperator{\Coker}{Coker}%
\DeclareMathOperator{\Sel}{Sel}%
\DeclareMathOperator{\ord}{ord}%
\DeclareMathOperator{\new}{new}%
\DeclareMathOperator{\an}{an}%

\newcommand{\abcd}[4]{\left(
        \begin{smallmatrix}#1&#2\\#3&#4\end{smallmatrix}\right)}

\usepackage{graphicx}
\setlength{\parskip}{1mm}

\begin{document}
\mbox{}
\large
\begin{enumerate}

  
\item A nursing researcher is working with people who have had a particular type of major surgery.  This researcher proposes that people will recover from the operation more quickly if friends and family are in the room with them for the first 48 hours after the operation.  It is known that time to recover from this kind of surgery is normally distributed with a mean of 12 days and a standard deviation of 5 days.  The procedure of having friends and family in the room for the period after the surgery is tried with a randomly selected patient.  This patient recovers in 4 days.

\begin{enumerate}
	\item Using the 0.01 significance level, what should the researcher conclude?  \\[4in]
	\item Explain your answer.
\end{enumerate}

\newpage

\item A researcher predicts that listening to classical music while solving math problems will make a particular brain area more active. To test this, a research participant has her brain scanned while listening to classical music and solving math problems, and the brain area of interest has a percent signal change of 5.8.  From many previous studies with this same math-problems procedure (but not listening to music), it is known that the signal change in this brain area is normally distributed with a mean of 3.5 and a standard deviation of 1.0.  

\begin{enumerate}
	\item Using the 0.05 level, what should the researcher conclude?\\[4in]
	\item Explain your answer.

\end{enumerate}

\end{enumerate}

\end{document}
