\documentclass[12pt]{article}%
\newcommand{\thisdocument}{}%
\newcommand{\thiscourse}{PSYC 3330: Elem Stats for Behav Sci}
\newcommand{\thissemester}{Week 7}
\usepackage{amsmath}%
\usepackage{amsfonts}%
\usepackage{amssymb}%
\usepackage{amsthm}%

%\textwidth = 6.5 in
%\textheight = 8 in
%\oddsidemargin = 0.0 in
%\evensidemargin = 0.0 in
%\topmargin = 0.0 in
%\headheight = 0.0 in
%\headsep = 0.0 in
\parskip = 0.1in
\parindent = 0.0in

\usepackage[left=1in,right=1in, top=1in, bottom=0.5in]{geometry}

\usepackage{fancyhdr,ifthen}
\pagestyle{fancy}
\lfoot{}  % no footers (in pagestyle fancy)
\cfoot{}
% running left heading
\lhead{\bfseries\Large\em \noindent{}\hspace{-.2em}\thiscourse{}\\[1mm]}
% running right heading
%\newcommand{\spc}{1.31em}
\newcommand{\spc}{1em}

\rhead{\em \sf Tarleton State University \hfill \sf \thisdocument{}}

\setlength{\headheight}{3ex}
\newcommand{\mainhead}[1]{\begin{center}{\Large \bf #1}\end{center}}
\newcommand{\head}[1]{\vspace{1.5ex}\par\noindent{\large \bf #1}\par\noindent}
\newcommand{\subhead}[1]{\vspace{2ex}\par\noindent{\sl #1}\vspace{1ex}\par\noindent{}}
\newcommand{\ptitle}{\sl}

\newcommand{\hra}{\hookrightarrow}


%%%% Theoremstyles
\theoremstyle{plain}
\newtheorem{theorem}{Theorem}[section]
\newtheorem{proposition}[theorem]{Proposition}
\newtheorem{corollary}[theorem]{Corollary}
\newtheorem{claim}[theorem]{Claim}
\newtheorem{lemma}[theorem]{Lemma}
\newtheorem{conjecture}[theorem]{Conjecture}

\theoremstyle{definition}
\newtheorem{definition}[theorem]{Definition}
\newtheorem{algorithm}[theorem]{Algorithm}
\newtheorem{question}[theorem]{Question}
\newtheorem{problem}[theorem]{Problem}
\newtheorem{goal}[theorem]{Goal}

\theoremstyle{remark}
\newtheorem{remark}[theorem]{Remark}
\newtheorem{remarks}[theorem]{Remarks}
\newtheorem{example}[theorem]{Example}
\newtheorem{exercise}[theorem]{Exercise}


\DeclareMathOperator{\SO}{SO}%
\DeclareMathOperator{\Sp}{Sp}%
\DeclareMathOperator{\SL}{SL}%
\DeclareMathOperator{\End}{End}%
\DeclareMathOperator{\Tr}{Tr}%
\DeclareMathOperator{\Res}{Res}%
\DeclareMathOperator{\res}{res}%
\DeclareMathOperator{\BSD}{BSD}%
\DeclareMathOperator{\Gal}{Gal}%
\DeclareMathOperator{\GL}{GL}%
\DeclareMathOperator{\Aut}{Aut}%
\DeclareMathOperator{\Reg}{Reg}%
\DeclareMathOperator{\Vis}{Vis}%
\DeclareMathOperator{\Ker}{Ker}%
\DeclareMathOperator{\Coker}{Coker}%
\DeclareMathOperator{\Sel}{Sel}%
\DeclareMathOperator{\ord}{ord}%
\DeclareMathOperator{\new}{new}%
\DeclareMathOperator{\an}{an}%

\newcommand{\abcd}[4]{\left(
        \begin{smallmatrix}#1&#2\\#3&#4\end{smallmatrix}\right)}

\usepackage{graphicx}
\setlength{\parskip}{1mm}

\begin{document}
\mbox{}
\large
\begin{enumerate}

\item A researcher is interested in whether people are able to identify emotions correctly in other people when they are extremely tired.  It is known that accuracy ratings of people in the general population (who are not extremely tired) are normally distributed with a mean of 82 and a standard deviation of 8.  In the present study, however, the researcher arranges to test 50 people who had no sleep the previous night.  The mean accuracy for these 50 individuals was 78.  What should the researcher conclude?

\newpage

\item A researcher is interested in the conditions that affect the number of dreams per month that people report in which they are alone.  We will assume that based on extensive previous research, it is known that in the general population, the number of such dreams per month follows a normal curve, with $\mu=5$ and $\sigma=4$.  The researcher wants to test the prediction that the number of such dreams will be greater among people who have recently experienced a traumatic event.  Thus, the researcher studies 36 individuals who have recently experienced a traumatic event, having them keep a record of their dreams for a month.  Their mean number of alone dreams is 8.  Should you conclude that people who have recently had a traumatic experience have a significantly different number of dreams in which they are alone?


\end{enumerate}

\end{document}
