% Created 2017-08-22 Tue 11:00
\documentclass[11pt]{article}
\usepackage[utf8]{inputenc}
\usepackage[T1]{fontenc}
\usepackage{fixltx2e}
\usepackage{graphicx}
\usepackage{longtable}
\usepackage{float}
\usepackage{wrapfig}
\usepackage{rotating}
\usepackage[normalem]{ulem}
\usepackage{amsmath}
\usepackage{textcomp}
\usepackage{marvosym}
\usepackage{wasysym}
\usepackage{amssymb}
\usepackage{hyperref}
\tolerance=1000
\date{August 28, 2017}
\title{Week 1 lecture notes - PSYC 5316}
\hypersetup{
  pdfkeywords={},
  pdfsubject={},
  pdfcreator={Emacs 25.2.1 (Org mode 8.2.10)}}
\begin{document}

\maketitle

\section*{Course outline}
\label{sec-1}

\begin{enumerate}
\item Review of classical statistical methods (5 weeks)
\begin{itemize}
\item Basic probability
\item distributions used for applied work
\item sampling distributions and confidence intervals
\item hypothesis testing
\item common hypothesis tests (including t-test, anova, chi-square, etc.)
\end{itemize}
\item Robust methods (3 weeks)
\begin{itemize}
\item bootstrapping
\item robust measures of location (including trimmed means, Winsorized means, $M$-estimators, etc.)
\item inferences based on robust measures
\end{itemize}
\item Bayesian methods (5 weeks)
\begin{itemize}
\item Bayes' Theorem, priors, likelihoods, and posteriors
\item estimating proportions and rates 
\begin{itemize}
\item exact methods via conjugate priors
\item approximate methods, using Markov chain Monte Carlo (MCMC)
\end{itemize}
\item fitting models with JAGS and R
\item Bayesian hypothesis testing
\end{itemize}
\end{enumerate}


\section*{Basic definitions}
\label{sec-2}
\subsection*{probability function}
\label{sec-2-1}
\subsection*{expected value and variance}
\label{sec-2-2}
\subsection*{conditional probability and independence}
\label{sec-2-3}

\section*{Distributions}
\label{sec-3}
\subsection*{Binomial}
\label{sec-3-1}
\subsection*{Normal}
\label{sec-3-2}
% Emacs 25.2.1 (Org mode 8.2.10)
\end{document}