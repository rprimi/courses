\documentclass[10pt]{article}
\usepackage[left=1in,right=1in,top=1in,bottom=1in]{geometry}
\usepackage{amsmath}
\usepackage{fancyhdr}
\lhead{PSYC 5316}
\chead{Homework 5}
\cfoot{}
\rhead{Fall 2017}

\pagestyle{fancy}

\begin{document}

\noindent
In this assignment, you will use R to download a large data set, explore the data, develop publication-quality plots, and perform some statistical tests.  In particular, you will explore data from Experiment 2 of one of my recent papers:

\begin{itemize}
\item Faulkenberry, T. J., Cruise, A., Lavro, D., \& Shaki, S. (2016). Response trajectories capture the continuous dynamics of the size congruity effect. {\it Acta Psychologica, 163}, 114-123. doi:10.1016/j.actpsy.2015.11.010
\end{itemize}

\noindent
For the questions below, compose your work in an R script, and submit your completed R script to me by email.  Please use comments to indicate the questions you are answering.

\begin{enumerate}
\item The data is available at the following URL:  Load the data set into a data frame called \verb|rawdata|.

\item As you will read in section 1.2.1 (page 116), we split response times into two components: initiation time (init), which is the time required to begin moving the computer mouse, and movement time (MT), which is the elapsed time of mouse movement.  Thus, $RT=init+MT$.  However, the computer software only recorded Init and RT for every trial.  Use \verb|mutate| to add the variable \verb|MT| to your dataset, and call this new data frame \verb|data|.

\item Use the ``split-apply-combine'' paradigm to reproduce Table 2 (page 118).  Note: you only need to do this for \verb|MT| and \verb|init|, since \verb|AUC| is not included in the dataset for this assignment.

\item Use \verb|ggplot| to reproduce Figure 3 and Figure 4 (page 119).  Error bars are not required, but I'll give you extra points if you can figure out how to construct them.  Can you figure out how to combine the two plots into ONE plot?  (hint: use faceting).

\item Construct plots of initiation time that mirror those in Figures 3 and 4 for movement time.  Do you think I should have included them in the published paper?  Why/why not?

\item Fit the ANOVA models for movement time reported in section 2.2.1 (Time analyses).  Note that I ran separate models for leftward and rightward trajectories.  Make sure your summaries match the ones reported (they should!).

\item Fit ANOVA models for initiation time, separated by response side.  Are there any significant effects?  

  
\end{enumerate}  
\end{document}