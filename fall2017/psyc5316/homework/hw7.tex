\documentclass[10pt]{article}
\usepackage[left=1in,right=1in,top=1in,bottom=1in]{geometry}
\usepackage{amsmath}
\usepackage{fancyhdr}
\lhead{PSYC 5316}
\chead{Homework 7}
\cfoot{}
\rhead{Fall 2017}

\pagestyle{fancy}

\begin{document}

\begin{enumerate}

\item Consider the data set given in the table below:

  \begin{table}[h!]
    \centering
    \begin{tabular}{cccccc}
      \hline
      $x$ & 5 & 8 & 9 & 7 & 14\\
      $y$ & 3 & 1 & 6 & 7 & 19\\
      \hline
    \end{tabular}
  \end{table}
  \begin{enumerate}
  \item Use the \verb|lm| function in R to find the equation of the least-squares regression line $y=a+bx$
  \item Plot the data and regression line together
    \item Construct a density plot of the residuals.  Comment on the overall fit of the model.
  \end{enumerate}


\item Consider the following data relating GPA to SAT score:
\begin{table}[h!]
    \centering
    \begin{tabular}{ccccccccc}
      \hline
      SAT & 500 & 530 & 590 & 660 & 610 & 700 & 570 & 640\\
      GPA & 2.3 & 3.1 & 2.6 & 3.0 & 2.4 & 3.3 & 2.6 & 3.5\\
      \hline
    \end{tabular}
  \end{table}
  \begin{enumerate}
  \item Assume a linear model $SAT = a+b\cdot GPA$, and compute maximum likelihood estimates for the parameters $a$ and $b$.
  \item Based on your model, what $SAT$ score would you predict for someone with a $GPA$ of 3.2?
  \item Construct 95\% confidence intervals for the parameters $a$ and $b$.
    \item Based on your 95\% CIs, can you conclude that $GPA$ is a significant predictor of $SAT$?  Explain.
  \end{enumerate}
  

\item This exercise illustrates the importance of looking at your data before assuming that it is linear.  Consider the following data:

  \begin{table}[h!]
    \centering
    \begin{tabular}{cccccc}
      \hline
      $x$ & 1 & 2 & 3 & 4 & 5\\
      $y$ & 0.74 & 2.22 & 6.04 & 16.20 & 44.55\\
      \hline
    \end{tabular}
  \end{table}

  \begin{enumerate}
  \item Construct a scatter plot of the data.  Does it look linear?
    \item Use \verb|lm| to fit a linear model $y = a+bx$.  
    \item Construct a new scatter plot of $x$ versus $\log y$.  Describe the shape. (Hint: just type \verb|plot(x,log(y))| in the R console).  What do you notice?
    \item Use \verb|lm| to fit a linear model for $\log(y) = a+bx$
      \item Construct residual plots for both models.  Based on these plots, which do you think is the better model?  Explain.
  \end{enumerate}

\item This exercise illustrates a different method of linear model fit called ``least absolute value'' regression.  Consider the following data:

  \begin{table}[h!]
    \centering
    \begin{tabular}{cccccc}
      \hline
      $x$ & 1 & 2 & 3 & 4 & 5\\
      $y$ & 5.25 &  10.12 &  15.40 & 18.55 & 202.12\\
      \hline
    \end{tabular}
  \end{table}

  \begin{enumerate}
  \item Construct a scatter plot of the data.  What do you notice?
  \item Use \verb|lm| to fit a linear model $y=a+bx$.
  \item Instead of minimizing the squared residuals (as \verb|lm| does), lets try minimizing the \textit{absolute value} of the residuals.  Modify the code on line 39 of \verb|week7.R| (the code from the lecture) to compute the {\it absolute value} of the residuals instead of the square of the residuals.
  \item Using \verb|optim|, find the parameters $a$ and $b$ that minimize the absolute value of the residuals.
  \item Using \verb|abline|, add both regression lines to your scatter plot.  Which is the better fit?
    \item Based on this exercise, when do you think using ``least absolute value'' regression might be most useful?
  \end{enumerate}
  
\end{enumerate}  
\end{document}