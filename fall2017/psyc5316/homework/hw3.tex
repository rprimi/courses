\documentclass[10pt]{article}
\usepackage[left=1in,right=1in,top=1in,bottom=1in]{geometry}
\usepackage{amsmath}
\usepackage{fancyhdr}
\lhead{PSYC 5316}
\chead{Homework 3}
\cfoot{}
\rhead{Fall 2017}

\pagestyle{fancy}

\begin{document}

\begin{enumerate}

\item Assume a sample of $n=25$ is randomly selected from a normal distribution with $\sigma=5$.  Suppose you get a sample mean of $\overline{x}=45$.  What is the 95\% confidence interval for $\mu$?

\item A manufacturer claims that its light bulbs have an average life span of $\mu = 1200$ hours, with a standard deviation of $\sigma=25$.  If you randomly test 36 light bulbs and find that their average life span is $\overline{x}=1150$, does a 95\% confidence interval for $\mu$ suggest that the claim $\mu=1200$ is unreasonable?  Explain.

\item Recall that a confidence interval $\mu$ (with known $\sigma$) can be found from the equation
  \[
    \Biggl(\overline{x}-c\frac{\sigma}{\sqrt{n}}, \overline{x}+c\frac{\sigma}{\sqrt{n}}\Biggr)
  \]

  \noindent
  What values of $c$ would be needed to compute 80\%, 92\%, and 98\% confidence intervals, respectively?

\item Suppose $n=16$, $\sigma=2$, and $\mu=30$.  Assume normality and determine
  \begin{enumerate}
  \item $p(\overline{x}<29)$
  \item $p(\overline{x}>30.5)$
    \item $p(29 < \overline{x} < 31)$
  \end{enumerate}

\item Someone claims that within a certain neighborhood, the average cost of a house is $\mu=100,000$ dollars with a standard deviation of $\sigma=10,000$ dollars.  Suppose that based on $n=16$ homes, you find that the average cost of a house is $\overline{x}=95,000$ dollars.  Assuming normality, what is the probability of getting a sample mean this low (or lower) if the claims about the mean and standard deviation are true?

\item Compute a 95\% confidence interval if:
  \begin{enumerate}
  \item $n=10$, $\overline{x}=26$, $s=9$
  \item $n=18$, $\overline{x}=132$, $s=20$
    \item $n=25$, $\overline{x}=52$, $s=12$
    \end{enumerate}


\item Repeat Exercise 6, but compute 99\% confidence intervals instead.

  \item Rats are subjected to a drug that might affect aggression.  Suppose that for a random sample of rats, measures of aggression are found to be
    \[
      5, 12, 23, 24, 18, 9, 18, 11, 36, 15.
    \]

    \noindent
    Compute a 95\% confidence interval for the mean, assuming that the scores are from a normal distribution.

  \item Explain the meaning of a 95\% confidence interval to someone who has never had a course in statistics.

    \item Last week, we discovered that for a normal model, the maximum likelihood estimate for the population mean $\mu$ is the sample mean $\overline{x}$.  Based on our work this week, explain what happens to the {\it precision} of our MLE as sample size increases. (Hint: what is precision? How would we compute it?)
\end{enumerate}
\end{document}