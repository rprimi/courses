% Created 2017-01-04 Wed 16:34
\documentclass[10pt]{article}
\usepackage[utf8]{inputenc}
\usepackage[T1]{fontenc}
\usepackage{fixltx2e}
\usepackage{graphicx}
\usepackage{longtable}
\usepackage{float}
\usepackage{wrapfig}
\usepackage{rotating}
\usepackage[normalem]{ulem}
\usepackage{amsmath}
\usepackage{textcomp}
\usepackage{marvosym}
\usepackage{wasysym}
\usepackage{amssymb}
\usepackage{hyperref}
\tolerance=1000
\usepackage[left=1in,right=1in,bottom=1in,top=1in]{geometry}
\author{Thomas J. Faulkenberry, Ph.D.}
\date{Spring 2017}
\title{PSYC 5301: Research Methods}
\hypersetup{
  pdfkeywords={},
  pdfsubject={},
  pdfcreator={Emacs 25.1.1 (Org mode 8.2.10)}}
\begin{document}

\maketitle

\section*{Contact info}
\label{sec-1}
\begin{itemize}
\item Professor: Thomas J. Faulkenberry, Ph.D
\item Office: Math 319
\item Office hours: MWF 9-11, TR 1-3
\item Email: faulkenberry@tarleton.edu
\item Twitter: \href{http://twitter.com/tomfaulkenberry}{tomfaulkenberry}
\item Phone: 254-968-9816
\end{itemize}

\section*{Course description}
\label{sec-2}

This course is designed to provide the student with a solid grounding in
the techniques of experimentation and subsequent statistical modeling that
form the empirical basis of modern psychological science.  We will 
accomplish this through lectures, textbook reading, and several hands-on
"laboratory" experiences, each designed to give the student a taste of the
research process, including data collection, analysis, and reporting.
All students enrolled in this course are required to have 
\textbf{previously taken} PSYC 5300 (Behavioral Statistics). 

\section*{Course materials}
\label{sec-3}

\begin{itemize}
\item \emph{The Design and Conduct of Meaningful Experiments Involving Human Participants: 25 Scientific Principles} by R. B. Bausell \href{https://www.amazon.com/Conduct-Meaningful-Experiments-Involving-Participants/dp/0199385238}{Amazon link}
\item \emph{APA Publication Manual} (6th ed.) \href{http://www.amazon.com/Publication-Manual-American-Psychological-Association/dp/1433805618/}{Amazon link}
\item JASP statistical software (free download from \href{http://jasp-stats.org}{jasp-stats.org})
\end{itemize}

\section*{NEED TO UPDATE Student learning outcomes}
\label{sec-4}

\begin{enumerate}
\item Develop a research question and write hypotheses appropriate for it
\item Design and conduct a research project to test hypotheses
\item Analyze data collected from research using computer software
\item Communicate the findings of research as a complete APA style manuscript
\end{enumerate}

\section*{NEED TO UPDATE Requirements and grading}
\label{sec-5}
\begin{itemize}
\item online quizzes
\begin{itemize}
\item Quiz 1 - structure of scientific knowledge
\item Quiz 2 - APA style
\item Quiz 3 - Research ethics
\item Quiz 4 - Variables and measurement
\item Quiz 5 - Experimental control
\end{itemize}
\item labs (100 points)
\begin{itemize}
\item 5 labs at 20 points each
\end{itemize}
\item IRB protocol
\item class participation
\end{itemize}

\section*{Course Communication}
\label{sec-6}

I am your primary resource for this course. I AM an experimental psychologist, so I do the stuff I teach about daily. Hence, my primary interest is for you to learn this material and do well in the course. You may contact me using any means necessary (email and Twitter are the best). That being said, many people prefer to use Blackboard messages. I don’t mind these, but keep in mind that I may not receive your message until I actually open Blackboard. With email/Twitter, if you send me a message at 9:00 pm, I may actually respond pretty quickly. 

\section*{University Policy on "F" Grades}
\label{sec-7}

Beginning in Fall 2015, Tarleton will begin differentiating between a failed grade in a class because a student never attended (F0 grade), stopped attending at some point in the semester (FX grade), or because the student did not pass the course (F) but attended the entire semester. These grades will be noted on the official transcript. Stopping or never attending class can result in the student having to return aid monies received.  For more information see the Tarleton Financial Aid website.

\section*{Academic Honesty}
\label{sec-8}

Cheating, plagiarism (submitting another person’s materials or ideas as one’s own), or doing work for another person who will receive academic credit are all disallowed. This includes the use of unauthorized books, notebooks, or other sources in order to secure of give help during an examination, the unauthorized copying of examinations, assignments, reports, or term papers, or the presentation of unacknowledged material as if it were the student’s own work. Disciplinary action may be taken beyond the academic discipline administered by the faculty member who teaches the course in which the cheating took place.

In particular, any exam taken online must be completed without the aid of any unauthorized resource (including using any search engine, Google, etc.).  Authorized resources are limited only to the official textbook and any lecture notes from the course.  Any other authorized resources will be provided to you before the exam.  The minimum sanction for violation of this policy is a grade of 0 on the affected exam.

\section*{Academic Affairs Core Value Statements}
\label{sec-9}

\subsection*{Academic Integrity Statement}
\label{sec-9-1}
Tarleton State University's core values are integrity, leadership, tradition, civility, excellence, and service.  Central to these values is integrity, which is maintaining a high standard of personal and scholarly conduct.  Academic integrity represents the choice to uphold ethical responsibility for one’s learning within the academic community, regardless of audience or situation.

\subsection*{Academic Civility Statement}
\label{sec-9-2}
Students are expected to interact with professors and peers in a respectful manner that enhances the learning environment. Professors may require a student who deviates from this expectation to leave the face-to-face (or virtual) classroom learning environment for that particular class session (and potentially subsequent class sessions) for a specific amount of time. In addition, the professor might consider the university disciplinary process (for Academic Affairs/Student Life) for egregious or continued disruptive behavior.

\subsection*{Academic Excellence Statement}
\label{sec-9-3}
Tarleton holds high expectations for students to assume responsibility for their own individual learning. Students are also expected to achieve academic excellence by:
\begin{itemize}
\item honoring Tarleton’s core values, upholding high standards of habit and behavior.
\item maintaining excellence through class attendance and punctuality, preparing for active participation in all learning experiences.
\item putting forth their best individual effort.
\item continually improving as independent learners.
\item engaging in extracurricular opportunities that encourage personal and academic growth.
\item reflecting critically upon feedback and applying these lessons to meet future challenges.
\end{itemize}

\section*{Students with Disabilities Policy}
\label{sec-10}

It is the policy of Tarleton State University to comply with the Americans with Disabilities Act and other applicable laws. If you are a student with a disability seeking accommodations for this course, please contact the Center for Access and Academic Testing, at 254.968.9400 or caat@tarleton.edu. The office is located in Math 201. More information can be found at www.tarleton.edu/caat or in the University Catalog.



\textbf{\textbf{Note:  any changes to this syllabus will be communicated to you by the instructor!}}

\section*{Schedule of lectures}
\label{sec-11}
\begin{center}
\begin{tabular}{rll}
Week & Date & Topic\\
\hline
1 & 1/17 & Lab: why randomization matters / Intro to JASP\\
3 & 1/31 & Lab: independent groups experiment - theory and design\\
5 & 2/14 & Lab: independent groups experiment - data analysis\\
7 & 2/28 & Lab: repeated measures experiment - theory and design\\
9 & 3/21 & Lab: repeated measures experiment - data analysis\\
11 & 4/4 & Lab: factorial experiment (between-subjects)\\
13 & 4/18 & Lab: factorial experiment - data analysis\\
15 & 5/2 & Lab: factorial experiment with repeated measures\\
\end{tabular}
\end{center}
% Emacs 25.1.1 (Org mode 8.2.10)
\end{document}