% Created 2017-01-04 Wed 10:15
\documentclass[10pt]{article}
\usepackage[utf8]{inputenc}
\usepackage[T1]{fontenc}
\usepackage{fixltx2e}
\usepackage{graphicx}
\usepackage{longtable}
\usepackage{float}
\usepackage{wrapfig}
\usepackage{rotating}
\usepackage[normalem]{ulem}
\usepackage{amsmath}
\usepackage{textcomp}
\usepackage{marvosym}
\usepackage{wasysym}
\usepackage{amssymb}
\usepackage{hyperref}
\tolerance=1000
\usepackage[left=1in,right=1in,bottom=1in,top=1in]{geometry}
\author{tom}
\date{Spring 2017}
\title{PSYC 3330: Elementary Statistics for the Behavioral Sciences}
\hypersetup{
  pdfkeywords={},
  pdfsubject={},
  pdfcreator={Emacs 25.1.1 (Org mode 8.2.10)}}
\begin{document}

\maketitle

\section*{Contact info}
\label{sec-1}
\begin{itemize}
\item Professor: Thomas J. Faulkenberry, Ph.D
\item Office: Math 319
\item Office hours: MTWRF 1:30-3:00 pm; TR 10:00-11:30 am
\item Email: faulkenberry@tarleton.edu
\item Twitter: \href{http://twitter.com/tomfaulkenberry}{@tomfaulkenberry}
\item Phone: 254-968-9816
\end{itemize}

\section*{Course description}
\label{sec-2}

Statistical methods are the primary tool for research in psychology.  
They are what allow us as researchers to make consistent, data-driven 
decisions.  As such, this is an extremely important course and one that I 
take very seriously as your professor.

The topics we will cover this semester will include descriptive statistics 
(how we describe data) and inferential statistics (how we make decisions 
about data).  Specifically, this includes central tendency, variability, 
correlation, the distinction between populations and samples, hypothesis 
testing, statistical significance, and a variety of inferential tests 
that we can apply to data, including t-tests and analysis of variance.

\section*{Course materials}
\label{sec-3}
\begin{itemize}
\item \emph{Statistics for the Behavioral Sciences (10th ed.)} by Gravetter and Wallnau \href{http://www.amazon.com/Statistics-Behavioral-Sciences-MindTap-Psychology/dp/1305504917/}{Amazon link}
\item Note:  older editions of this textbook are just fine.  Please feel free to find a used copy of an older edition on Amazon; it will save you a LOT of money!
\end{itemize}

\section*{Student learning outcomes}
\label{sec-4}
\begin{enumerate}
\item Identify variables under study (including independent and dependent variables)
\item Choose appropriate measures of descriptive statistics
\item Select and perform appropriate inferential statistics
\item Draw appropriate statistical conclusions from results of analyses
\end{enumerate}

\section*{Requirements and grading}
\label{sec-5}
\begin{itemize}
\item Exams (450 pts)
\item Unit quizzes (100 pts)
\item Homework exercises (50 pts)
\item \emph{Total = 600 points}
\end{itemize}

Grades will be assigned based on the percentage of points you accumulate out of these 600 points.  I will use the standard grading scale of A=90\%, B=80\%, etc.

\subsection*{Exams (75\% of grade)}
\label{sec-5-1}
There will be four total exams throughout the semester, occurring 
approximately once every three to four weeks.  They will cover material 
from lectures, quizzes, and homework exercises.  Exam questions will be a mix of multiple choice and short answer.  Exams are due by 11:59 pm on 
Sunday of the given week.  Each exam will have a time limit and may only 
be attempted once.

Due dates:

\begin{itemize}
\item Exam 1 (Feb 12 at 11:59 pm)
\item Exam 2 (Mar 12 at 11:59 pm)
\item Exam 3 (Apr 23 at 11:59 pm)
\item Final exam (April 30 at 11:59 pm)
\end{itemize}

\subsection*{Unit quizzes (17\% of grade)}
\label{sec-5-2}
At the end of each unit, you will complete a quiz over the content of that 
unit. Each quiz counts for 10 possible points.  Since there are 10 units, 
you will earn up to 100 points for your quiz grade.

\subsection*{Homework exercises (8\% of grade)}
\label{sec-5-3}
In order to practice the statistical concepts you learn this semester, you
will complete a short online homework assignment for each unit.  Each 
homework assignment will be assigned on Blackboard, and you will be given
unlimited attempts to do each problem.  Most assignments will contain 
around 5 problems and be worth a maximum of 5 points.  Since there are 10
units, you will be able to earn a maximum of 50 points for the semester.

\section*{Course Communication}
\label{sec-6}
I am your primary resource for this course. I AM an experimental 
psychologist, so I do the stuff I teach about daily. Hence, my primary 
interest is for you to learn this material and do well in the course. You 
may contact me using any means necessary (email is the best).  That being 
said, many people prefer to use Blackboard messages. I don’t mind these, 
but keep in mind that I may not receive your message until I actually 
open Blackboard and check those messages.  I check emails much more 
frequently and tend to respond more quickly to those.

\section*{University Policy on "F" Grades}
\label{sec-7}
Beginning in Fall 2015, Tarleton will begin differentiating between a 
failed grade in a class because a student never attended (F0 grade), 
stopped attending at some point in the semester (FX grade), or because 
the student did not pass the course (F) but attended the entire semester. 
These grades will be noted on the official transcript. Stopping or never 
attending class can result in the student having to return aid monies 
received.  For more information see the Tarleton Financial Aid website.

\section*{Academic Honesty}
\label{sec-8}
Cheating, plagiarism (submitting another person’s materials or ideas as 
one’s own), or doing work for another person who will receive academic 
credit are all disallowed. This includes the use of unauthorized books, 
notebooks, or other sources in order to secure of give help during an 
examination, the unauthorized copying of examinations, assignments, 
reports, or term papers, or the presentation of unacknowledged material 
as if it were the student’s own work. Disciplinary action may be taken 
beyond the academic discipline administered by the faculty member who 
teaches the course in which the cheating took place.

In particular, any exam taken online must be completed without the aid of 
any unauthorized resource (including using any search engine, Google,
etc.).  Authorized resources are limited only to the official textbook 
and any lecture notes from the course.  Any other authorized resources 
will be provided to you before the exam.  The minimum sanction for 
violation of this policy is a grade of 0 on the affected exam.

\section*{Students with Disabilities Policy}
\label{sec-9}
It is the policy of Tarleton State University to comply with the Americans
with Disabilities Act (ADA) and other federal, state, and local laws 
relative to the provision of disability services. Students with 
disabilities attending Tarleton State University may contact the Office 
of Disability Services at (254) 968-9478 to request appropriate 
accommodation. Furthermore, formal accommodation requests cannot be made 
until the student has been officially admitted to Tarleton State 
University.

\textbf{Note:  any changes to this syllabus will be communicated to you by the instructor!}

\section*{Semester Schedule}
\label{sec-10}
\begin{center}
\begin{tabular}{rll}
Unit & Dates & Topic\\
\hline
1 & Jan 16-22 & Displaying data\\
2 & Jan 23-29 & Descriptives 1: central tendency, variation, and z-scores\\
3 & Jan 30-Feb 5 & Descriptives 2: correlation\\
 & Feb 6-12 & \textbf{Exam 1 (due February 12)}\\
4 & Feb 13-19 & The normal distribution: measuring likelihood\\
5 & Feb 20-26 & The logic of hypothesis testing\\
6 & Feb 27-Mar 5 & Testing means of samples of \textbf{known} populations: $z$-tests\\
 & Mar 6-12 & \textbf{Exam 2 (due March 12)}\\
 & Mar 13-19 & \emph{Spring break!}\\
7 & Mar 20-26 & Testing means of samples of \textbf{unknown} populations: $t$-tests\\
8 & Mar 27-Apr 2 & More $t$-tests (independent samples, etc.)\\
9 & Apr 3-9 & Analysis of variance (ANOVA): one independent variable\\
10 & Apr 10-16 & Analysis of variance (ANOVA): two independent variables\\
 & Apr 17-23 & \textbf{Exam 3 (due April 23)}\\
 & Apr 24-30 & \textbf{Final exam (due April 30)}\\
\end{tabular}
\end{center}
% Emacs 25.1.1 (Org mode 8.2.10)
\end{document}