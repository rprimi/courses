% Created 2017-01-04 Wed 16:33
\documentclass[10pt]{article}
\usepackage[utf8]{inputenc}
\usepackage[T1]{fontenc}
\usepackage{fixltx2e}
\usepackage{graphicx}
\usepackage{longtable}
\usepackage{float}
\usepackage{wrapfig}
\usepackage{rotating}
\usepackage[normalem]{ulem}
\usepackage{amsmath}
\usepackage{textcomp}
\usepackage{marvosym}
\usepackage{wasysym}
\usepackage{amssymb}
\usepackage{hyperref}
\tolerance=1000
\usepackage[left=1in,right=1in,bottom=1in,top=1in]{geometry}
\date{Spring 2017}
\title{PSYC 3435: Principles of Research for the Behavorial Sciences}
\hypersetup{
  pdfkeywords={},
  pdfsubject={},
  pdfcreator={Emacs 25.1.1 (Org mode 8.2.10)}}
\begin{document}

\maketitle

\section*{Contact info}
\label{sec-1}
\begin{itemize}
\item Professor: Thomas J. Faulkenberry, Ph.D
\item Office: Math 319
\item Office hours: MTWRF 1-3 pm
\item Email: faulkenberry@tarleton.edu
\item Website: \url{http://tomfaulkenberry.github.io}
\item Phone: 254-968-9816
\end{itemize}

\section*{Course description}
\label{sec-2}

This laboratory course is designed to introduce students to the philosophy of 
scientific inquiry with an emphasis on experimental methodology. This will be 
accomplished via a combination of traditional lectures along with an application 
of principles through laboratory experimentation and demonstration. Essentially, 
we will cover the "nuts and bolts" of putting together and completing a research 
project in psychology. To this end, we will cover all fourteen chapters of the 
textbook. All students enrolled in this course are required to have 
\textbf{previously taken} PSYC 3309 (Writing in Psychology) and PSYC 3330 (Statistics). 

\section*{Course materials}
\label{sec-3}

\begin{itemize}
\item \emph{The Process of Research in Psychology} (3rd ed.) by McBride (2015) \href{https://www.amazon.com/Process-Research-Psychology-Dawn-McBride/dp/1483347605/}{Amazon link}
\item \emph{APA Publication Manual} (6th ed.) \href{http://www.amazon.com/Publication-Manual-American-Psychological-Association/dp/1433805618/}{Amazon link}
\item JASP statistical software (free download from \href{http://jasp-stats.org}{jasp-stats.org}
\end{itemize}

\section*{Student learning outcomes}
\label{sec-4}

\begin{enumerate}
\item Describe advantages of the scientific method compared to other approaches
\item Identify various research designs and uses of each
\item Conduct library research
\item Paraphrase and cite material from primary sources
\item Develop a research question and write hypotheses appropriate for it
\item Design and conduct a research project to test hypotheses
\item Analyze data collected from research using computer software
\item Communicate the findings of research as a complete APA style manuscript
\end{enumerate}

\section*{Requirements and grading}
\label{sec-5}

There will be four exams throughout the semester.  They will cover material from the text and the lectures.  Exam questions will be a mix of multiple choice and short answer and will be worth 100 points each.

Dates:

\begin{itemize}
\item Exam 1 (Friday, Feb 17)
\item Exam 2 (Friday, March 31)
\item Exam 3 (Friday, April 28)
\item Final exam (Monday, May 8, 3:00-5:30 pm)
\end{itemize}

\subsection*{Unit quizzes (14\% of grade)}
\label{sec-5-1}

Every Friday (excluding exam weeks), there will be a short in-class quiz over the material covered in lectures on Monday and Wednesday. Typically, the quizzes will consist of multiple choice questions, although the format may change occasionally depending on material.  Your final quiz average will be scaled to a total of 100 points for the semester (i.e., if you average 8.6/10 points over all quizzes, your final quiz grade will be 86/100).

\subsection*{Lab assignments (14\% of grade)}
\label{sec-5-2}

There will be four lab assignments this semester.  Each assignment is designed to give you first hand experience with essential research skills, such as conducting an experiment, collecting data, and writing up an APA manuscript.  Except for Lab 1 (an APA exercise), each of you will administer a short experiment, collect some data, analyze the collected data for the class, and compose an APA manuscript based on the experiment and the results. 

Lab Assignments and dates

\begin{itemize}
\item Lab 1 – APA formatting (10 points), due Friday, Feb 10
\item Lab 2.1 – Correlational design, data collection (10 points), due Friday, Feb 17
\item Lab 2.2 – Correlational design, analysis and report (20 points), due Friday, Mar 3
\item Lab 3.1 – Independent groups design, data collection (10 points), due Friday, Mar 10
\item Lab 3.2 – Independent groups design, analysis and report (20 points), due Friday, Apr 7
\item Lab 4.1 – Factorial design, data collection (10 points), due Friday, Apr 14
\item Lab 4.2 – Factorial design, analysis and report (20 points), due Wednesday, May 5
\end{itemize}

\subsection*{Research participation (8\% of grade)}
\label{sec-5-3}

Research is the lifeblood of psychology!  In order to promote your growth as a psychological scientist, you will be required to participate in 2 "units" of research over the course of the semester.  Roughly defined, a "unit" of research consists of either (1) participation in one hour of psychological experiments, usually offered right here in the Psychology department at Tarleton, or (2) a 3-5 page critical reflection of an empirical research article related to human learning and/or cognition, usually chosen by me.  Instructions for participation will be given in class.

\section*{Course Communication}
\label{sec-6}

I am your primary resource for this course. I AM an experimental psychologist, so I do the stuff I teach about daily. Hence, my primary interest is for you to learn this material and do well in the course. If you have a question, always feel free to stop by my office and visit.  If you require electronic communication, email is best, but you may also send me a Blackboard message.  Just keep in mind that I don't get any notifications of Blackboard messages, so I may not see your message until I next log into the course.

\section*{University Policy on "F" Grades}
\label{sec-7}

Beginning in Fall 2015, Tarleton will begin differentiating between a failed grade in a class because a student never attended (F0 grade), stopped attending at some point in the semester (FX grade), or because the student did not pass the course (F) but attended the entire semester. These grades will be noted on the official transcript. Stopping or never attending class can result in the student having to return aid monies received.  For more information see the Tarleton Financial Aid website.

\section*{Academic Honesty}
\label{sec-8}

Tarleton State University expects its students to maintain high standards of
personal and scholarly conduct. Students guilty of academic dishonesty are
subject to disciplinary action. Cheating, plagiarism (submitting another person’s materials or ideas as one’s own), or doing work for another person who will receive academic credit are all disallowed. This includes the use of unauthorized books, notebooks, or other sources in order to secure of give help during an examination, the unauthorized copying of examinations, assignments, reports, or term papers, or the presentation of unacknowledged material as if it were the student’s own work. Disciplinary action may be taken beyond the academic discipline administered by the faculty member who teaches the course in which the cheating took place.

In particular, any exam taken online must be completed without the aid of any unauthorized resource (including using any search engine, Google, etc.).  Authorized resources are limited only to the official textbook and any lecture notes from the course.  Any other authorized resources will be provided to you before the exam.  The minimum sanction for violation of this policy is a grade of 0 on the affected exam.

Each student’s honesty and integrity are taken for granted. However, if I find
evidence of academic misconduct I will pursue the matter
to the fullest extent permitted by the university. ACADEMIC MISCONDUCT OR
DISHONESTY WILL RESULT IN A GRADE OF F FOR THE COURSE.  Students are
strongly advised to avoid even the \emph{appearance} of academic misconduct. 

\section*{Academic Affairs Core Value Statements}
\label{sec-9}

\subsection*{Academic Integrity Statement}
\label{sec-9-1}
Tarleton State University's core values are integrity, leadership, tradition, civility, excellence, and service.  Central to these values is integrity, which is maintaining a high standard of personal and scholarly conduct.  Academic integrity represents the choice to uphold ethical responsibility for one’s learning within the academic community, regardless of audience or situation.

\subsection*{Academic Civility Statement}
\label{sec-9-2}
Students are expected to interact with professors and peers in a respectful manner that enhances the learning environment. Professors may require a student who deviates from this expectation to leave the face-to-face (or virtual) classroom learning environment for that particular class session (and potentially subsequent class sessions) for a specific amount of time. In addition, the professor might consider the university disciplinary process (for Academic Affairs/Student Life) for egregious or continued disruptive behavior.

\subsection*{Academic Excellence Statement}
\label{sec-9-3}
Tarleton holds high expectations for students to assume responsibility for their own individual learning. Students are also expected to achieve academic excellence by:
\begin{itemize}
\item honoring Tarleton’s core values, upholding high standards of habit and behavior.
\item maintaining excellence through class attendance and punctuality, preparing for active participation in all learning experiences.
\item putting forth their best individual effort.
\item continually improving as independent learners.
\item engaging in extracurricular opportunities that encourage personal and academic growth.
\item reflecting critically upon feedback and applying these lessons to meet future challenges.
\end{itemize}

\section*{Students with Disabilities Policy}
\label{sec-10}

It is the policy of Tarleton State University to comply with the Americans
with Disabilities Act and other applicable laws. If you are a student with a
disability seeking accommodations for this course, please contact Trina
Geye, Director of Student Disability Services, at 254.968.9400 or
geye@tarleton.edu. Student Disability Services is
located in Math 201. More information can be found at www.tarleton.edu/sds or in the University Catalog.


\textbf{\textbf{Note:  any changes to this syllabus will be communicated to you by the instructor!}}

\section*{Schedule of lectures}
\label{sec-11}

\begin{center}
\begin{tabular}{rll}
Week & Dates & Lecture topic\\
\hline
1 & Jan 16-20 & Knowing in psychological science (Ch 1)\\
2 & Jan 23-27 & Reading the literature / APA style (Ch 2,8)\\
3 & Jan 30-Feb 3 & Basic research methods (Ch 3)\\
4 & Feb 6-10 & Ethics in psychological science (Ch 5)\\
5 & Feb 13-17 & \textbf{Exam 1}\\
6 & Feb 20-24 & Experiments: selecting and manipulating variables (Ch 4)\\
7 & Feb 27-Mar 3 & Experiments: sampling methods (Ch 6)\\
8 & Mar 6-10 & Experiments: basic designs (Ch 11)\\
- & Mar 13-17 & \textbf{Spring break!}\\
9 & Mar 20-24 & Experiments: factorial designs (Ch 11)\\
10 & Mar 27-31 & \textbf{Exam 2}\\
11 & Apr 3-7 & Non-experiments: survey methods (Ch 9)\\
12 & Apr 10-14 & Non-experiments: correlations and regression (Ch 10)\\
13 & Apr 17-21 & Non-experiments: quasi-experiments and developmental designs (Ch 12,13)\\
14 & Apr 24-28 & \textbf{Exam 3}\\
15 & May 1-5 & Course review\\
16 & May 8-12 & \textbf{Final exam: Monday, May 8, 3:00-5:30 pm}\\
\end{tabular}
\end{center}
% Emacs 25.1.1 (Org mode 8.2.10)
\end{document}