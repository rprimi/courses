\documentclass{article}
%\usepackage[letterpaper, left=1in,right=1in,top=1in,bottom=1in]{geometry}
\usepackage{tikz}
\newcommand{\thisdocument}{Introductory measurement activity}
\newcommand{\thiscourse}{PSYC 4301: Psychological Tests and Measurements}
%\newcommand{\thissemester}{Spring 2020}
\usepackage{amsmath}%
\usepackage{amsfonts}%
\usepackage{amssymb}%
\usepackage{amsthm}%

%\textwidth = 6.5 in
%\textheight = 8 in
%\oddsidemargin = 0.0 in
%\evensidemargin = 0.0 in
%\topmargin = 0.0 in
%\headheight = 0.0 in
%\headsep = 0.0 in
\parskip = 0.1in
\parindent = 0.0in

\usepackage[left=1in,right=1in, top=1in, bottom=0.5in]{geometry}

\usepackage{fancyhdr,ifthen}
\pagestyle{fancy}
\lfoot{}  % no footers (in pagestyle fancy)
\cfoot{}
% running left heading
\lhead{\bfseries\Large\em \noindent{}\hspace{-.2em}\thiscourse{}\\[1mm]}
% running right heading
%\newcommand{\spc}{1.31em}
\newcommand{\spc}{1em}

\rhead{\em \sf Tarleton State University \hfill \sf \thisdocument{}}

\setlength{\headheight}{3ex}
\newcommand{\mainhead}[1]{\begin{center}{\Large \bf #1}\end{center}}
\newcommand{\head}[1]{\vspace{1.5ex}\par\noindent{\large \bf #1}\par\noindent}
\newcommand{\subhead}[1]{\vspace{2ex}\par\noindent{\sl #1}\vspace{1ex}\par\noindent{}}
\newcommand{\ptitle}{\sl}

\newcommand{\hra}{\hookrightarrow}


%%%% Theoremstyles
\theoremstyle{plain}
\newtheorem{theorem}{Theorem}[section]
\newtheorem{proposition}[theorem]{Proposition}
\newtheorem{corollary}[theorem]{Corollary}
\newtheorem{claim}[theorem]{Claim}
\newtheorem{lemma}[theorem]{Lemma}
\newtheorem{conjecture}[theorem]{Conjecture}

\theoremstyle{definition}
\newtheorem{definition}[theorem]{Definition}
\newtheorem{algorithm}[theorem]{Algorithm}
\newtheorem{question}[theorem]{Question}
\newtheorem{problem}[theorem]{Problem}
\newtheorem{goal}[theorem]{Goal}

\theoremstyle{remark}
\newtheorem{remark}[theorem]{Remark}
\newtheorem{remarks}[theorem]{Remarks}
\newtheorem{example}[theorem]{Example}
\newtheorem{exercise}[theorem]{Exercise}


\DeclareMathOperator{\SO}{SO}%
\DeclareMathOperator{\Sp}{Sp}%
\DeclareMathOperator{\SL}{SL}%
\DeclareMathOperator{\End}{End}%
\DeclareMathOperator{\Tr}{Tr}%
\DeclareMathOperator{\Res}{Res}%
\DeclareMathOperator{\res}{res}%
\DeclareMathOperator{\BSD}{BSD}%
\DeclareMathOperator{\Gal}{Gal}%
\DeclareMathOperator{\GL}{GL}%
\DeclareMathOperator{\Aut}{Aut}%
\DeclareMathOperator{\Reg}{Reg}%
\DeclareMathOperator{\Vis}{Vis}%
\DeclareMathOperator{\Ker}{Ker}%
\DeclareMathOperator{\Coker}{Coker}%
\DeclareMathOperator{\Sel}{Sel}%
\DeclareMathOperator{\ord}{ord}%
\DeclareMathOperator{\new}{new}%
\DeclareMathOperator{\an}{an}%

\newcommand{\abcd}[4]{\left(
        \begin{smallmatrix}#1&#2\\#3&#4\end{smallmatrix}\right)}



%\pagestyle{empty}104 105 180

\begin{document}
\mbox{}
\vspace{1cm}

Using the tiles on the next page, how many tiles would you need to purchase in order to tile this space?  Use the provided ruler to find out!

\begin{center}
\begin{tikzpicture}[x=1mm,y=1mm]
  \draw [line width=0.3mm] (0,0) -- (125,1);
  \draw [line width=0.3mm] (125,1) -- (126,90);
  \draw [line width=0.3mm] (126,90) -- (0,90);
  \draw [line width=0.3mm] (0,90) -- (0,0);
\end{tikzpicture}
\end{center}


\newpage

\newdimen\spaceleft
\spaceleft=\dimexpr\textheight-\pagetotal-14pt\relax
\pgfmathsetmacro{\gridWidth}{\textwidth - mod(\textwidth,1.8cm)}
\pgfmathsetmacro{\gridHeight}{\spaceleft - mod(\spaceleft,1.8cm)}

\begin{tikzpicture}
    \draw [step=1.8cm,thick] (0,0) grid (\gridWidth pt,\gridHeight pt);
\end{tikzpicture}


\newpage
\mbox{}
\vspace{1cm}

\begin{tikzpicture}[rotate=-50, transform shape, x=1cm, y=1cm]
  \draw (-0.25,0) rectangle (21.5,3); % draw ruler shape

  \draw (0.05,2.22) node[right] {cm}; % draw cm label
  \draw (20.2,0.78) node[right,rotate=180] {in}; % draw in label
  \draw (1, 1.5) node[right] {\textsc{Acme Ruler Company}};
  
  %% (centimeter marks)
  \foreach \x in {0,1,...,4}{
    \draw (\x,3) -- (\x,2.5) node[below]{\x};
  }
  \foreach \x in {5,6,...,21}{ % set the repeated index!!!
    \pgfmathsetmacro{\newindex}{int(\x-1)}
    \draw (\x,3) -- (\x,2.5) node[below]{\newindex};
  }
  %% millimeter marks
  \foreach \x in {0.1,0.2,...,21}{
    \draw (\x,3) -- (\x,2.7);
  }

  %% 5mm marks
  \foreach \x in {0.5,1,...,21}{
    \draw (\x,3) -- (\x,2.6);
  }

  %% inch marks
  \foreach \x in {0,1,...,8}{
    \pgfmathsetmacro{\reverseinch}{int(8-\x)}
    \draw (\x in,0) -- (\x in,0.6) node[below, rotate=180]{\reverseinch};
  }

  %% 1/16 inch marks
  \foreach \x in {0.0625,0.125,...,8}{
    \draw (\x in,0) -- (\x in,0.2);
  }

  %% 1/8 inch marks
  \foreach \x in {0.125,0.250,...,8}{
    \draw (\x in,0) -- (\x in,0.3);
    }

  %% 1/4 inch marks
  \foreach \x in {0.250,0.500,...,8}{
    \draw (\x in,0) -- (\x in,0.4);
  }

  %% 1/2 inch marks
  \foreach \x in {0.5,1,...,8}{
    \draw (\x in,0) -- (\x in,0.5);
    }
  
  

\end{tikzpicture}

\end{document}
  
