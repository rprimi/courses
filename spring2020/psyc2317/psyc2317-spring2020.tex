% Created 2020-01-13 Mon 11:35
% Intended LaTeX compiler: pdflatex
\documentclass[10pt]{article}
\usepackage[utf8]{inputenc}
\usepackage[T1]{fontenc}
\usepackage{graphicx}
\usepackage{grffile}
\usepackage{longtable}
\usepackage{wrapfig}
\usepackage{rotating}
\usepackage[normalem]{ulem}
\usepackage{amsmath}
\usepackage{textcomp}
\usepackage{amssymb}
\usepackage{capt-of}
\usepackage{hyperref}
\usepackage[left=1in,right=1in,bottom=1in,top=1in]{geometry}
\date{Spring 2020}
\title{PSYC 2317: Statistical Methods for Psychology}
\hypersetup{
 pdfauthor={},
 pdftitle={PSYC 2317: Statistical Methods for Psychology},
 pdfkeywords={},
 pdfsubject={},
 pdfcreator={Emacs 26.2 (Org mode 9.1.9)}, 
 pdflang={English}}
\begin{document}

\maketitle

\section*{Contact info}
\label{sec:org4a538aa}
\begin{itemize}
\item Professor: Thomas J. Faulkenberry, Ph.D
\item Office: Math 319
\item Office hours: MF 8-11 am; MWF 1-3pm (\emph{or by appointment})
\item Email: faulkenberry@tarleton.edu
\item Website: \url{http://tomfaulkenberry.github.io}
\item Phone: 254-968-9816
\end{itemize}

\section*{Course description}
\label{sec:orgab4571d}

Statistical methods are the primary tool for research in psychology. They are what allow us as researchers to make consistent, data-driven decisions.  As such, this is an extremely important course and one that I take very seriously as your professor. The topics we will cover this semester will include descriptive statistics (how we describe data) and inferential statistics (how we make decisions about data).  Specifically, this includes central tendency, variability, the distinction between populations and samples, estimation and hypothesis testing, and a variety of inferential techniques that we can apply to data, including \(t\)-tests and analysis of variance.

\begin{itemize}
\item \emph{Textbook}: There are many very good textbooks on statistical methods (e.g., Howell, Gravetter, etc.)  You can pick up used copies from Amazon for relatively low cost). However, I do not lecture directly from any specific textbook. I do recommend the following as a good reference to follow this semester, as I think it is a readable textbook that does a good job of bridging theory with hands-on, practical examples.
\begin{itemize}
\item \emph{Learning Statistics with JASP: A Tutorial for Psychology Students and Other Beginners}, by Navarro, Foxcroft, and Faulkenberry (2019), freely downloadable from \href{http://learnstatswithjasp.com}{www.learnstatswithjasp.com}
\end{itemize}

\item \emph{Calculator}: you will need a basic scientific calculator to perform various calculations on homework and exams. I use a TI-84 (mine is over 20 years old and still works!), but any calculator with square roots and logarithms will suffice.

\item \emph{JASP statistical software}: some of the exercises we do this semester will be done on the computer, and we will use JASP, which is freely downloadable from \href{http://www.jasp-stats.com}{www.jasp-stats.com}. JASP should be installed on all campus computers, but if you have a personal computer/laptop, go ahead and install it there\ldots{}it is free, so why not?
\end{itemize}

\section*{Student learning outcomes}
\label{sec:orga89da76}
\begin{enumerate}
\item Compute appropriate measures of descriptive statistics from data
\item Calculate probabilities for normally distributed data
\item Understand sampling distributions and their relationship to statistical inference
\item Estimate population parameters using confidence intervals
\item Understand the basic structure of hypothesis tests
\item Select and perform appropriate hypothesis tests for given data sets
\end{enumerate}

\section*{Requirements and grading}
\label{sec:orgd2e433a}
\begin{itemize}
\item Exam 1 (100 pts)
\item Exam 2 (100 pts)
\item Exam 3 (100 pts)
\item Final exam (100 pts)
\item Weekly quizzes (100 pts)
\item Homework exercises (100 pts)
\item \emph{Total = 600 points}
\end{itemize}

Grades will be assigned based on the percentage of points you accumulate out of these 600 points. I will use the standard grading scale of A=90\%, B=80\%, etc.

\subsection*{Exams}
\label{sec:org42491f0}
There will be four total exams throughout the semester, occurring approximately once every three to four weeks.  They will cover material from lectures, quizzes, and homework exercises. Exams are closed book with short answer questions (you'll a Bluebook for each exam) and will be completed in class.

Exam dates:

\begin{itemize}
\item Exam 1 (Thursday, February 20)
\item Exam 2 (Thursday, March 26)
\item Exam 3 (Thursday, April 23)
\item Final exam (Friday, May 1, 8-10:00 am)
\end{itemize}

\subsection*{Weekly quizzes}
\label{sec:orgf643401}
I will administer a short in-class multiple choice quiz at the beginning of class on Tuesday of each week (excluding Tuesdays immediately following exam weeks). The quiz will cover content from the previous week's lecture and homework exercises. Each quiz counts for 10 possible points.  There will be 10 of these quizzes, so you will earn up to 100 points for your overall quiz grade.

\subsection*{Homework exercises}
\label{sec:org2e49550}
In order to practice the statistical concepts you learn this semester, you will complete a short homework assignment every week. A set of homework exercises (usually between 5 and 10 problems) will be provided to you each week. You may work collaboratively on the homework exercises, but any work submitted must reflect your own understanding of the material (in other words, don't just copy someone else's work to submit). Completed exercises should be handwritten neatly on clean paper. Each homework assignment will be due at the beginning of class on Tuesday of the week after it is assigned.

\section*{Course Communication}
\label{sec:org0f0c890}

Email is the primary means of official communication for this course.  If you have questions about the course, always feel free to send me an email at faulkenberry@tarleton.edu.  I only ask that you adhere to two guidelines:
\begin{itemize}
\item please include the course number (PSYC 2317) in the subject line.  For example, one good way to do this is:  Subject: [PSYC 2317] Question about Exam 2
\item please use proper email etiquette.  Include a salutation (e.g., Dear Dr. Faulkenberry), complete sentences, and a closing (e.g., "Regards, Your Name").  You might be surprised how many times I get an email from a nondescript email address with no indication from WHOM the email was sent!
\end{itemize}

Also, I will send periodic class announcements via email.  Thus, it is imperative that you check your \emph{Tarleton email address} regularly so that you don't miss any of these messages.

\section*{CV Points for Psychology Majors}
\label{sec:orgca4da3c}
All Tarleton psychology majors will be required to accumulate a certain number of "CV points" as a pass/fail component of their senior capstone course. CV is an acronym for "curriculum vitae", which is the traditional name of an academic resume. Because it is a requirement of senior capstone, no psychology major will be able to graduate without completion/verification of the required 15 CV points. Some classes may build in CV point assignments, but ultimately it is the students’ responsibility to monitor their participation and acquire points during their time at Tarleton. More information on pre-approved CV points, submission, and tracking of these points can be found in the CV Point Canvas site. Please note that submissions are graded, and may not be approved for points if they do not meet the CV standard.  If a student has a question about CV points, they should send an email to psychcvpointga@tarleton.edu.

\section*{University Policy on "F" Grades}
\label{sec:orgdecdaee}
Beginning in Fall 2015, Tarleton will begin differentiating between a failed grade in a class because a student never attended (F0 grade), stopped attending at some point in the semester (FX grade), or because the student did not pass the course (F) but attended the entire semester. These grades will be noted on the official transcript. Stopping or never attending class can result in the student having to return aid monies received.  For more information see the Tarleton Financial Aid website.

\section*{Academic Honesty}
\label{sec:orgedd8b60}

Tarleton State University expects its students to maintain high standards of personal and scholarly conduct. Students guilty of academic dishonesty are subject to disciplinary action. Cheating, plagiarism (submitting another person’s materials or ideas as one’s own), or doing work for another person who will receive academic credit are all disallowed. This includes the use of unauthorized books, notebooks, or other sources in order to secure of give help during an examination, the unauthorized copying of examinations, assignments, reports, or term papers, or the presentation of unacknowledged material as if it were the student’s own work. Disciplinary action may be taken beyond the academic discipline administered by the faculty member who teaches the course in which the cheating took place.

In particular, any exam taken online must be completed without the aid of any unauthorized resource (including using any search engine, Google, etc.).  Authorized resources are limited only to the official textbook and any lecture notes from the course.  Any other authorized resources will be provided to you before the exam.  The minimum sanction for violation of this policy is a grade of 0 on the affected exam.

Each student’s honesty and integrity are taken for granted. However, if I find evidence of academic misconduct I will pursue the matter to the fullest extent permitted by the university. ACADEMIC MISCONDUCT OR DISHONESTY WILL RESULT IN A GRADE OF F FOR THE COURSE.  Students are strongly advised to avoid even the \emph{appearance} of academic misconduct. 

\section*{Academic Affairs Core Value Statements}
\label{sec:org2c5beda}
\subsection*{Academic Integrity Statement}
\label{sec:orgfd6d00c}
Tarleton State University's core values are integrity, leadership, tradition, civility, excellence, and service.  Central to these values is integrity, which is maintaining a high standard of personal and scholarly conduct.  Academic integrity represents the choice to uphold ethical responsibility for one’s learning within the academic community, regardless of audience or situation.

\subsection*{Academic Civility Statement}
\label{sec:org4e57e71}
Students are expected to interact with professors and peers in a respectful manner that enhances the learning environment. Professors may require a student who deviates from this expectation to leave the face-to-face (or virtual) classroom learning environment for that particular class session (and potentially subsequent class sessions) for a specific amount of time. In addition, the professor might consider the university disciplinary process (for Academic Affairs/Student Life) for egregious or continued disruptive behavior.

\subsection*{Academic Excellence Statement}
\label{sec:org4270937}
Tarleton holds high expectations for students to assume responsibility for their own individual learning. Students are also expected to achieve academic excellence by:
\begin{itemize}
\item honoring Tarleton’s core values, upholding high standards of habit and behavior.
\item maintaining excellence through class attendance and punctuality, preparing for active participation in all learning experiences.
\item putting forth their best individual effort.
\item continually improving as independent learners.
\item engaging in extracurricular opportunities that encourage personal and academic growth.
\item reflecting critically upon feedback and applying these lessons to meet future challenges.
\end{itemize}

\section*{Students with Disabilities Policy}
\label{sec:org40afc27}

It is the policy of Tarleton State University to comply with the Americans with Disabilities  Act (www.ada.gov) and other applicable laws.  If you are a student with a disability seeking accommodations for this course, please contact the Center for Access and Academic Testing, at 254.968.9400 or caat@tarleton.edu. The office is located in Math 201. More information can be found at www.tarleton.edu/caat or in the University Catalog.​

\textbf{Note:  any changes to this syllabus will be communicated to you by the instructor!}

\section*{Semester Schedule}
\label{sec:orgdaedd78}
\begin{center}
\begin{tabular}{rll}
Unit & Dates & Topic\\
\hline
 & Jan 13-17 & (no class -- I will be at Joint Mathematics Meetings)\\
1 & Jan 20-24 & Measures of central tendency\\
2 & Jan 27-31 & Measures of variability\\
3 & Feb 3-7 & The normal distribution\\
4 & Feb 10-14 & Distributions of sample means\\
 & \textbf{Feb 17-21} & \textbf{Exam 1}\\
5 & Feb 24-28 & Estimating with confidence intervals\\
6 & Mar 2-6 & Testing hypotheses\\
 & Mar 9-13 & (no class -- Spring Break)\\
7 & Mar 16-20 & Introduction to the \(t\)-test\\
 & \textbf{Mar 23-27} & \textbf{Exam 2}\\
8 & Mar 30 - Apr 3 & \(t\)-tests for independent samples\\
9 & Apr 6-10 & Confidence intervals for \(t\)-tests\\
10 & Apr 13-17 & Analysis of variance\\
 & \textbf{Apr 20-24} & \textbf{Exam 3}\\
 & Apr 28 & Course review\\
 & \textbf{May 1} & \textbf{Final exam - Friday, May 1, 8-10 am}\\
\end{tabular}
\end{center}
\end{document}