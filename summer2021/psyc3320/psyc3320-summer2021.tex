% Created 2021-04-29 Thu 10:04
% Intended LaTeX compiler: pdflatex
\documentclass[10pt]{article}
\usepackage[utf8]{inputenc}
\usepackage[T1]{fontenc}
\usepackage{graphicx}
\usepackage{grffile}
\usepackage{longtable}
\usepackage{wrapfig}
\usepackage{rotating}
\usepackage[normalem]{ulem}
\usepackage{amsmath}
\usepackage{textcomp}
\usepackage{amssymb}
\usepackage{capt-of}
\usepackage{hyperref}
\usepackage[left=1in,right=1in,bottom=1in,top=1in]{geometry}
\date{Summer 2021}
\title{PSYC 3320: Psycholinguistics}
\hypersetup{
 pdfauthor={},
 pdftitle={PSYC 3320: Psycholinguistics},
 pdfkeywords={},
 pdfsubject={},
 pdfcreator={Emacs 27.2 (Org mode 9.4.4)}, 
 pdflang={English}}
\begin{document}

\maketitle

\section*{Contact info}
\label{sec:orgd368cfa}
\begin{itemize}
\item Professor: Thomas J. Faulkenberry, Ph.D
\item Office: Math 319
\item Office hours: MTWRF 9-11 am (or by appointment)
\item Email: faulkenberry@tarleton.edu
\item Website: \url{http://tomfaulkenberry.github.io}
\item Phone: 254-968-9816
\end{itemize}

\section*{Course description}
\label{sec:org090fcd0}

(from catalog) \emph{The course emphasizes the study of language, understanding languages, producing language and speech, language development, and related topics such as reading, language and the brain, linguistic diversity, and universals.}

Psycholinguistics is an area of specialization within cognitive psychology that focuses on the psychological mechanisms of language processing.  This course provides an introduction to the field of psycholinguistics. Psycholinguistics is the field that studies the information processing mechanisms that govern the use of language in comprehension and production, acquisition, and representation.  A common misconception is that one learns about the specifics of particular languages in a course on psycholinguistics.  This is not the case.  Rather, like most of cognitive psychology, this course emphasizes the \emph{commonalities} that underlie human languages everywhere, and we will focus on what these commonalities tell us about the human ability to construct and understand language. 

\section*{Course materials}
\label{sec:orgf72cd41}
\begin{itemize}
\item \emph{The Psychology of Language: From Data to Theory (4th ed.)} by Harley \href{https://www.amazon.com/Psychology-Language-Data-Theory/dp/1848720890}{Amazon link}
\end{itemize}

\section*{Student learning outcomes}
\label{sec:orgae4b7e5}
\begin{enumerate}
\item Define the basic vocabulary of psycholinguistics
\item Discuss the biological bases of human communication.
\item Describe how humans perceive and produce speech.
\item Describe how sentences and discourse are processed and comprehended.
\item Discuss the acquisition of language.
\item Describe the psychological processes involved in reading.
\item Identify and explain problems in speech and language processes.
\item Explain bilingualism and second language acquisition.
\end{enumerate}

\section*{Requirements and grading}
\label{sec:orgce13eef}
\begin{itemize}
\item Exam 1 (100 pts)
\item Exam 2 (100 pts)
\item Exam 3 (100 pts)
\item Exam 4 (100 pts)
\item Unit quizzes (100 pts)
\item \emph{Total = 500 points}
\end{itemize}

Grades will be assigned based on the percentage of points you accumulate out of these 500 points.  I will use the standard grading scale of A=90\%, B=80\%, etc.

\subsection*{Exams (80\% of grade)}
\label{sec:orge5f8589}
There will be four total exams throughout the semester, occurring approximately once every 2  weeks.  They will cover material from your reading, online lectures, and quizzes.  Exam questions will be a mix of multiple choice and short answer.  Exams are due by 11:59 pm on the due date (see below).  Each exam will have a time limit and may only be attempted once.

Due dates:

\begin{itemize}
\item Exam 1 (Friday, May 28 at 11:59 pm)
\item Exam 2 (Friday, June 11 at 11:59 pm)
\item Exam 3 (Friday, June 25 at 11:59 pm)
\item Exam 4 (Wednesday, July 7 at 11:59 pm)
\end{itemize}

\subsection*{Unit quizzes (20\% of grade)}
\label{sec:orge2cf67e}
At the end of each unit, you will complete a quiz over the content of that  unit. Each quiz will be graded as number correct out of 10 possible points.  Your total quiz grade will be computed by scaling your average quiz percentage up to a 100 point score.  For example, if you average 8.6/10 on your unit quizzes, your total quiz grade for the semester will be 86/100. 

\section*{Course Communication}
\label{sec:orgbb70165}

Email is the primary means of communication for this course.  If you have questions about the course, always feel free to send me an email at faulkenberry@tarleton.edu.  I only ask that you adhere to two guidelines:
\begin{itemize}
\item please include the course number (PSYC 3320) in the subject line.  For example, one good way to do this is:  Subject: [PSYC 3320] Question about Exam 2
\item please use proper email etiquette.  Include a salutation (e.g., Dear Dr. Faulkenberry), complete sentences, and a closing (e.g., "Regards, Your Name").  You might be surprised how many times I get an email from a nondescript email address with no indication from WHOM the email was sent!
\end{itemize}

Also, I will be sending periodic emails to each of you that update you on course progress, due dates, etc. It is imperative that you check your \emph{Tarleton email address} regularly so that you don't miss any of these messages.

\section*{CV Points for Psychology Majors}
\label{sec:orgedff30d}
All Tarleton psychology majors will be required to accumulate a certain number of "CV points" as a pass/fail component of their senior capstone course. CV is an acronym for "curriculum vitae", which is the traditional name of an academic resume. Because it is a requirement of senior capstone, no psychology major will be able to graduate without completion/verification of the required 15 CV points. Some classes may build in CV point assignments, but ultimately it is the students’ responsibility to monitor their participation and acquire points during their time at Tarleton. More information on pre-approved CV points, submission, and tracking of these points can be found in the CV Point Canvas site. Please note that submissions are graded, and may not be approved for points if they do not meet the CV standard.  If a student has a question about CV points, they should send an email to psychcvpointga@tarleton.edu.

\section*{University Policy on "F" Grades}
\label{sec:org6093666}
Beginning in Fall 2015, Tarleton will begin differentiating between a 
failed grade in a class because a student never attended (F0 grade), 
stopped attending at some point in the semester (FX grade), or because 
the student did not pass the course (F) but attended the entire semester. 
These grades will be noted on the official transcript. Stopping or never 
attending class can result in the student having to return aid monies 
received.  For more information see the Tarleton Financial Aid website.

\section*{Academic Honesty}
\label{sec:org389e8b2}

Cheating, plagiarism (submitting another person’s materials or ideas as one’s own without proper attribution), or doing work for another person who will receive academic credit are all disallowed. This includes the use of unauthorized books, notebooks, or other sources in order to secure of give help during an examination, the unauthorized copying of examinations, assignments, reports, or term papers, or the presentation of unacknowledged material as if it were the student’s own work. Disciplinary action may be taken beyond the academic discipline administered by the faculty member who teaches the course in which the cheating took place.

In particular, any quiz or exam taken online must be completed without the aid of any unauthorized resource (including using any search engine, Google, etc.).  Authorized resources are limited only to the official textbook and any lecture notes from the course.  Any other authorized resources will be provided to you before the exam.  

The minimum sanction for \emph{any} act of academic dishonesty is a grade of 0 on the affected assignment; a grade of F for the course may be assigned in severe cases.

\section*{Academic Affairs Core Value Statements}
\label{sec:org9d502be}

\subsection*{Academic Integrity Statement}
\label{sec:orga1d8697}
Tarleton State University's core values are integrity, leadership, tradition, civility, excellence, and service.  Central to these values is integrity, which is maintaining a high standard of personal and scholarly conduct.  Academic integrity represents the choice to uphold ethical responsibility for one’s learning within the academic community, regardless of audience or situation.

\subsection*{Academic Civility Statement}
\label{sec:org7844214}
Students are expected to interact with professors and peers in a respectful manner that enhances the learning environment. Professors may require a student who deviates from this expectation to leave the face-to-face (or virtual) classroom learning environment for that particular class session (and potentially subsequent class sessions) for a specific amount of time. In addition, the professor might consider the university disciplinary process (for Academic Affairs/Student Life) for egregious or continued disruptive behavior.

\subsection*{Academic Excellence Statement}
\label{sec:org7c0aca3}
Tarleton holds high expectations for students to assume responsibility for their own individual learning. Students are also expected to achieve academic excellence by:
\begin{itemize}
\item honoring Tarleton’s core values, upholding high standards of habit and behavior.
\item maintaining excellence through class attendance and punctuality, preparing for active participation in all learning experiences.
\item putting forth their best individual effort.
\item continually improving as independent learners.
\item engaging in extracurricular opportunities that encourage personal and academic growth.
\item reflecting critically upon feedback and applying these lessons to meet future challenges.
\end{itemize}

\section*{Students with Disabilities Policy}
\label{sec:org9afde47}

It is the policy of Tarleton State University to comply with the Americans with Disabilities Act and other applicable laws. If you are a student with a disability seeking accommodations for this course, please contact Center for Access and Academic Testing at 254.968.9400 or caat@tarleton.edu or stop by Math 201. More information can be found at www.tarleton.edu/caat or in the University Catalog. 

\textbf{\textbf{Note:  any changes to this syllabus will be communicated to you by the instructor!}}

\section*{Semester Schedule}
\label{sec:orgf6c8d9f}

\begin{center}
\begin{tabular}{llr}
Unit & Topic & Book chapter(s)\\
\hline
1 (May 18-21) & The basics (cognitive psychology and linguistics) & 1,2\\
2 (May 24-28) & The foundations of language & 3\\
 & \textbf{Exam 1 (due Friday, May 28)} & \\
3 (May 31-June 4) & Language development & 4\\
4 (June 7-11) & Perception of words (visual and spoken) & 6,9\\
 & \textbf{Exam 2 (due Friday, June 11)} & \\
5 (June 14-18) & Reading & 7,8\\
6 (June 21-25) & Understanding the structure of sentences & 10\\
 & \textbf{Exam 3 (due Friday, June 25)} & \\
7 (June 28-July 2) & Word meaning & 11\\
8 (July 5-7) & Language production & 13\\
 & \textbf{Exam 4 (due Wednesday, July 7)} & \\
\end{tabular}
\end{center}
\end{document}